\documentclass[12pt]{article}
\usepackage[utf8]{inputenc}
\usepackage[margin=1in]{geometry}
\usepackage{amsmath}
\usepackage{amssymb}

\title{Machine Learning Notes - Introduction}
\author{}
\date{March 2019}

\begin{document}

\maketitle
%-------------------------------------------------
\section{Primer - Probability and Statistics}

\subsection{An introduction to Measure Theory}
\subsubsection*{$\sigma$ algebra}
\textbf{Definition}: Given a set $\Omega$, a $\sigma$-algebra on $Omega$ is a collection $\mathcal{A}$  \subset \:  $2^{\Omega}$ such that 
\begin{enumerate}
    \item $\mathcal{A}$ is non empty 
    \item $\mathcal{A}$ is closed under compliments i.e.\ if $E \in \mathcal{A} \implies E^{c} \in \mathcal{A}$
    \item $\mathcal{A}$ is closed under countable unions i.e.\ if $E_{1}, E_{2}... \in \mathcal{A} \implies \bigcup\limits_{i=1}^{\infty} E_{i} \in \mathcal{A}$
\end{enumerate}
\textbf{Remarks}
\begin{enumerate}
    \item $\Omega \in \mathcal{A}$ \: (Proof: $E \subset \Omega , \: E \in \mathcal{A} \implies E^{c} \in \mathcal{A} \implies E \: \cup \: E^{c} \in \mathcal{A} \implies \Omega \in \mathcal{A}$)
    \item $\phi \in \mathcal{A}$ \: (Proof: $\Omega \in \mathcal{A} \implies \Omega^{c} \in \mathcal{A} \implies \phi \in \mathcal{A}$)
    \item $\mathcal{A}$ is closed under countable intersections too i.e. $E_{1}, E_{2}... \in \mathcal{A} \implies \bigcap\limits_{i=1}^{\infty} E_{i} \in \mathcal{A}$
\end{enumerate}
\textbf{Examples}:
\begin{enumerate}
    \item If $\Omega$ = $\{0,1\} \implies 2^{\Omega} = \{ \phi,\: \{0\},\: \{1\},\: \{0,1\} \}$, where $2^{\Omega}$ is called the Power Set
    \item Let $\Omega = \{1,2,3,4\}$ , then $\mathcal{A} = \{ \phi, \{1,2,3,4\}, \{1,2\} , \{3,4\}\}$ is a $\sigma$ - algebra 
    \item $\mathcal{A} = \{\phi,\: \Omega\}$ is also a $\sigma$-algebra
    \item If $\Omega = \mathbb{R}$, the Borel $\sigma$ - algebra is $\mathcal{B} = \sigma(\mathcal{T})$ where $\mathcal{T}$ = all open sets of $\mathbb{R}$ . Any set $B \in \mathcal{B}$ is a Borel set
\end{enumerate}
\newpage
\subsubsection*{Measure}
\textbf{Definition}: A measure $\mu$ on $\Omega$ with $\sigma$-algebra $\mathcal{A}$ is a function $\mu : \mathcal{A} \xrightarrow{} [0,\infty]$ s.t.
\begin{enumerate}
    \item $\mu(\phi) = 0$
    \item Countable additivity - $\mu(\: \bigcup\limits_{i=1}^{\infty}E_{i} \:) = \sum\limits_{i=1}^{\infty} \mu(E_{i})$ where $E_1, E_2... \in \mathcal{A}$ are pairwise disjoint sets 
\end{enumerate}

\subsubsection*{Basic Properties Probability Measures}
A measure space is described by $(\Omega, \mathcal{A}, \mu)$. Similarly, a probability space is described by $(\Omega, \mathcal{A}, P)$ where $\Omega$ = sample space, $\mathcal{A}$ = $\sigma$-algebra, $P$ = Probability measure
\begin{enumerate}
    \item $P(\Omega) = 1$
    \item \textbf{Monotonicity}: If $E,F \in \mathcal{A}, E \subset F$ then $P(E) < P(F)$
    \item \textbf{Subadditivity}: If $E_{1}, E_{2}... \in \mathcal{A}$ then $P(\bigcup\limits_{i=1}^{\infty}E_i) \leq \sum\limits_{i=1}^{\infty} P(E_i)$
    \item \textbf{Continuity from above}: If $E_{1}, E_{2}... \in \mathcal{A}$ and $E_1 \subseteq E_2 \subseteq$... then $P(\: \bigcup\limits_{i=1}^{\infty}E_{i} \:) = \lim_{i\to\infty} P(E_i)$
    \item \textbf{Continuity from below}: If $E_1, E_2 .... \in \mathcal{A}$ and $E_1 \supseteq E_2 \supseteq$... \: and $P(E_1) < \infty$ then, \: $P(\bigcap\limits_{i=1}^{\infty}E_i) = \lim_{i \to\infty}P(E_i)$
\end{enumerate}

\subsubsection*{More properties of probability measures}
Let $E,F \in A$ then 
\begin{enumerate}
    \item $P(E \cup F) = P(E) + P(F) - P(E \cap F)$
    \item $P(E) = 1 - P(E^{c})$
    \item $P(E \cap F^{c}) = P(E) - P(E \cap F)$
    \item $P(\: \bigcup\limits_{i=1}^{n}E_{i} \:) \leq \sum\limits_{i=1}^{n}P(E_i)$ 
    \item Inclusion Exclusion Formula (Too long to write but important)  
\end{enumerate}

\subsubsection*{Random Variable}
\subsubsection*{PDF and CDF}
\subsubsection*{Expectation and Variance}
\subsubsection*{Poisson Distriubtion}
\subsubsection*{Exponential Distribution}
\subsubsection*{Hyper-geometric Distribution}
\subsubsection*{Normal Distribution}
\subsubsection*{Central Limit Theorem}
\subsubsection*{Hypothesis Testing and p-value}
\subsubsection*{Statistical Tests - $\chi, z, t, F$}
\subsubsection*{ANOVA}





%-------------------------------------------------
\section{Primer - Linear Algebra}

%-------------------------------------------------
\section{Primer - Other Useful Math}

%-------------------------------------------------
\section{Least Squares}

%-------------------------------------------------
\section{Nearest Neighbours}

%-------------------------------------------------
\section{Statistical Decision Theory}

%-------------------------------------------------
\section{Local Methods in Higher Dimensions}

%-------------------------------------------------
\section{Statistical Models, Supervised Learning and Function Approximation}
%-------------------------------------------------
\section{Structured Regression Models}
%-------------------------------------------------
\section{Classes of Restricted Estimators}
%-------------------------------------------------
\section{Bias-Variance Trade-off}
%-------------------------------------------------
\end{document}
